% Options for packages loaded elsewhere
\PassOptionsToPackage{unicode}{hyperref}
\PassOptionsToPackage{hyphens}{url}
%
\documentclass[
]{article}
\usepackage{amsmath,amssymb}
\usepackage{iftex}
\ifPDFTeX
  \usepackage[T1]{fontenc}
  \usepackage[utf8]{inputenc}
  \usepackage{textcomp} % provide euro and other symbols
\else % if luatex or xetex
  \usepackage{unicode-math} % this also loads fontspec
  \defaultfontfeatures{Scale=MatchLowercase}
  \defaultfontfeatures[\rmfamily]{Ligatures=TeX,Scale=1}
\fi
\usepackage{lmodern}
\ifPDFTeX\else
  % xetex/luatex font selection
\fi
% Use upquote if available, for straight quotes in verbatim environments
\IfFileExists{upquote.sty}{\usepackage{upquote}}{}
\IfFileExists{microtype.sty}{% use microtype if available
  \usepackage[]{microtype}
  \UseMicrotypeSet[protrusion]{basicmath} % disable protrusion for tt fonts
}{}
\makeatletter
\@ifundefined{KOMAClassName}{% if non-KOMA class
  \IfFileExists{parskip.sty}{%
    \usepackage{parskip}
  }{% else
    \setlength{\parindent}{0pt}
    \setlength{\parskip}{6pt plus 2pt minus 1pt}}
}{% if KOMA class
  \KOMAoptions{parskip=half}}
\makeatother
\usepackage{xcolor}
\usepackage[margin=1in]{geometry}
\usepackage{color}
\usepackage{fancyvrb}
\newcommand{\VerbBar}{|}
\newcommand{\VERB}{\Verb[commandchars=\\\{\}]}
\DefineVerbatimEnvironment{Highlighting}{Verbatim}{commandchars=\\\{\}}
% Add ',fontsize=\small' for more characters per line
\usepackage{framed}
\definecolor{shadecolor}{RGB}{248,248,248}
\newenvironment{Shaded}{\begin{snugshade}}{\end{snugshade}}
\newcommand{\AlertTok}[1]{\textcolor[rgb]{0.94,0.16,0.16}{#1}}
\newcommand{\AnnotationTok}[1]{\textcolor[rgb]{0.56,0.35,0.01}{\textbf{\textit{#1}}}}
\newcommand{\AttributeTok}[1]{\textcolor[rgb]{0.13,0.29,0.53}{#1}}
\newcommand{\BaseNTok}[1]{\textcolor[rgb]{0.00,0.00,0.81}{#1}}
\newcommand{\BuiltInTok}[1]{#1}
\newcommand{\CharTok}[1]{\textcolor[rgb]{0.31,0.60,0.02}{#1}}
\newcommand{\CommentTok}[1]{\textcolor[rgb]{0.56,0.35,0.01}{\textit{#1}}}
\newcommand{\CommentVarTok}[1]{\textcolor[rgb]{0.56,0.35,0.01}{\textbf{\textit{#1}}}}
\newcommand{\ConstantTok}[1]{\textcolor[rgb]{0.56,0.35,0.01}{#1}}
\newcommand{\ControlFlowTok}[1]{\textcolor[rgb]{0.13,0.29,0.53}{\textbf{#1}}}
\newcommand{\DataTypeTok}[1]{\textcolor[rgb]{0.13,0.29,0.53}{#1}}
\newcommand{\DecValTok}[1]{\textcolor[rgb]{0.00,0.00,0.81}{#1}}
\newcommand{\DocumentationTok}[1]{\textcolor[rgb]{0.56,0.35,0.01}{\textbf{\textit{#1}}}}
\newcommand{\ErrorTok}[1]{\textcolor[rgb]{0.64,0.00,0.00}{\textbf{#1}}}
\newcommand{\ExtensionTok}[1]{#1}
\newcommand{\FloatTok}[1]{\textcolor[rgb]{0.00,0.00,0.81}{#1}}
\newcommand{\FunctionTok}[1]{\textcolor[rgb]{0.13,0.29,0.53}{\textbf{#1}}}
\newcommand{\ImportTok}[1]{#1}
\newcommand{\InformationTok}[1]{\textcolor[rgb]{0.56,0.35,0.01}{\textbf{\textit{#1}}}}
\newcommand{\KeywordTok}[1]{\textcolor[rgb]{0.13,0.29,0.53}{\textbf{#1}}}
\newcommand{\NormalTok}[1]{#1}
\newcommand{\OperatorTok}[1]{\textcolor[rgb]{0.81,0.36,0.00}{\textbf{#1}}}
\newcommand{\OtherTok}[1]{\textcolor[rgb]{0.56,0.35,0.01}{#1}}
\newcommand{\PreprocessorTok}[1]{\textcolor[rgb]{0.56,0.35,0.01}{\textit{#1}}}
\newcommand{\RegionMarkerTok}[1]{#1}
\newcommand{\SpecialCharTok}[1]{\textcolor[rgb]{0.81,0.36,0.00}{\textbf{#1}}}
\newcommand{\SpecialStringTok}[1]{\textcolor[rgb]{0.31,0.60,0.02}{#1}}
\newcommand{\StringTok}[1]{\textcolor[rgb]{0.31,0.60,0.02}{#1}}
\newcommand{\VariableTok}[1]{\textcolor[rgb]{0.00,0.00,0.00}{#1}}
\newcommand{\VerbatimStringTok}[1]{\textcolor[rgb]{0.31,0.60,0.02}{#1}}
\newcommand{\WarningTok}[1]{\textcolor[rgb]{0.56,0.35,0.01}{\textbf{\textit{#1}}}}
\usepackage{graphicx}
\makeatletter
\def\maxwidth{\ifdim\Gin@nat@width>\linewidth\linewidth\else\Gin@nat@width\fi}
\def\maxheight{\ifdim\Gin@nat@height>\textheight\textheight\else\Gin@nat@height\fi}
\makeatother
% Scale images if necessary, so that they will not overflow the page
% margins by default, and it is still possible to overwrite the defaults
% using explicit options in \includegraphics[width, height, ...]{}
\setkeys{Gin}{width=\maxwidth,height=\maxheight,keepaspectratio}
% Set default figure placement to htbp
\makeatletter
\def\fps@figure{htbp}
\makeatother
\setlength{\emergencystretch}{3em} % prevent overfull lines
\providecommand{\tightlist}{%
  \setlength{\itemsep}{0pt}\setlength{\parskip}{0pt}}
\setcounter{secnumdepth}{-\maxdimen} % remove section numbering
\ifLuaTeX
  \usepackage{selnolig}  % disable illegal ligatures
\fi
\usepackage{bookmark}
\IfFileExists{xurl.sty}{\usepackage{xurl}}{} % add URL line breaks if available
\urlstyle{same}
\hypersetup{
  pdftitle={Tarea 1 - CA0307: Estadística Actuarial II},
  pdfauthor={Anthony Mauricio Jiménez Navarro \textbar{} C24067; Henri Gerard Gabert Hidalgo \textbar{} B93096; Juan Pablo Morgan Sandí \textbar{} C15319},
  hidelinks,
  pdfcreator={LaTeX via pandoc}}

\title{Tarea 1 - CA0307: Estadística Actuarial II}
\author{Anthony Mauricio Jiménez Navarro \textbar{} C24067 \and Henri
Gerard Gabert Hidalgo \textbar{} B93096 \and Juan Pablo Morgan Sandí
\textbar{} C15319}
\date{2024-10-22}

\begin{document}
\maketitle

{
\setcounter{tocdepth}{2}
\tableofcontents
}
\section{Librerias}\label{librerias}

\begin{Shaded}
\begin{Highlighting}[]
\FunctionTok{set.seed}\NormalTok{(}\DecValTok{2024}\NormalTok{)}
\end{Highlighting}
\end{Shaded}

\section{Ejercicio 1}\label{ejercicio-1}

Primero se crea la función a integrar, la cual sabemos que es \[
\int_0^1 \frac{e^{-x^2}}{1+x^2} d x
\]

\begin{Shaded}
\begin{Highlighting}[]
\NormalTok{f }\OtherTok{\textless{}{-}} \ControlFlowTok{function}\NormalTok{(x) \{}
  \FunctionTok{exp}\NormalTok{(}\SpecialCharTok{{-}}\NormalTok{x}\SpecialCharTok{\^{}}\DecValTok{2}\NormalTok{) }\SpecialCharTok{/}\NormalTok{ (}\DecValTok{1} \SpecialCharTok{+}\NormalTok{ x}\SpecialCharTok{\^{}}\DecValTok{2}\NormalTok{)}
\NormalTok{\}}
\end{Highlighting}
\end{Shaded}

Una vez hecho lo anterior, programos el algoritmo de Montecarlo.

\begin{Shaded}
\begin{Highlighting}[]
\CommentTok{\# Método de Montecarlo para aproximar la integral}
\NormalTok{montecarlo\_integration }\OtherTok{\textless{}{-}} \ControlFlowTok{function}\NormalTok{(N) \{}
  \CommentTok{\# Generamos N muestras aleatorias entre 0 y 1}
\NormalTok{  x }\OtherTok{\textless{}{-}} \FunctionTok{runif}\NormalTok{(N, }\DecValTok{0}\NormalTok{, }\DecValTok{1}\NormalTok{)}
  
  \CommentTok{\# Evaluamos la función en los puntos muestreados}
\NormalTok{  fx }\OtherTok{\textless{}{-}} \FunctionTok{f}\NormalTok{(x)}
  
  \CommentTok{\# Estimar la integral como el promedio de f(x)}
\NormalTok{  integral\_estimate }\OtherTok{\textless{}{-}} \FunctionTok{mean}\NormalTok{(fx)}

  \FunctionTok{return}\NormalTok{(integral\_estimate)}
\NormalTok{\}}

\NormalTok{N }\OtherTok{\textless{}{-}} \DecValTok{100000}  \CommentTok{\# número de muestras}

\NormalTok{montecarlo\_result }\OtherTok{\textless{}{-}} \FunctionTok{montecarlo\_integration}\NormalTok{(N)}

\FunctionTok{cat}\NormalTok{(}\StringTok{"Aproximación de la integral por Montecarlo:"}\NormalTok{, montecarlo\_result, }\StringTok{"}\SpecialCharTok{\textbackslash{}n}\StringTok{"}\NormalTok{)}
\end{Highlighting}
\end{Shaded}

\begin{verbatim}
## Aproximación de la integral por Montecarlo: 0.6189492
\end{verbatim}

Ahora, usando integrate.

\begin{Shaded}
\begin{Highlighting}[]
\NormalTok{integral\_exacta }\OtherTok{\textless{}{-}} \FunctionTok{integrate}\NormalTok{(f, }\DecValTok{0}\NormalTok{, }\DecValTok{1}\NormalTok{)}
\FunctionTok{cat}\NormalTok{(}\StringTok{"Aproximación de la integral por Integrate:"}\NormalTok{, integral\_exacta}\SpecialCharTok{$}\NormalTok{value, }\StringTok{"}\SpecialCharTok{\textbackslash{}n}\StringTok{"}\NormalTok{)}
\end{Highlighting}
\end{Shaded}

\begin{verbatim}
## Aproximación de la integral por Integrate: 0.618822
\end{verbatim}

\begin{Shaded}
\begin{Highlighting}[]
\FunctionTok{cat}\NormalTok{(}\StringTok{"El error absoluto de Integrate:"}\NormalTok{, integral\_exacta}\SpecialCharTok{$}\NormalTok{abs.error, }\StringTok{"}\SpecialCharTok{\textbackslash{}n}\StringTok{"}\NormalTok{)}
\end{Highlighting}
\end{Shaded}

\begin{verbatim}
## El error absoluto de Integrate: 6.870304e-15
\end{verbatim}

Ahora, la diferencia entre el resultado de Montecarlo y el de Integrate
es:

\begin{Shaded}
\begin{Highlighting}[]
\NormalTok{montecarlo\_result }\SpecialCharTok{{-}}\NormalTok{ integral\_exacta}\SpecialCharTok{$}\NormalTok{value}
\end{Highlighting}
\end{Shaded}

\begin{verbatim}
## [1] 0.0001272235
\end{verbatim}

\section{Ejercicio 2}\label{ejercicio-2}

Primero, se crea la función \(f_L\), la cual es \[
f_L(L)= \lambda e^{-\lambda L}
\] La cual, sabiendo que \(\lambda = 1\), es \(f_L(L)= e^{-L}\)

\begin{Shaded}
\begin{Highlighting}[]
\NormalTok{f\_L }\OtherTok{\textless{}{-}} \ControlFlowTok{function}\NormalTok{(L) \{}
  \FunctionTok{return}\NormalTok{(}\FunctionTok{exp}\NormalTok{(}\SpecialCharTok{{-}}\NormalTok{L)) }
\NormalTok{\}}
\end{Highlighting}
\end{Shaded}

Por el enunciado sabemos también que \(g(L) \sim N(3, 4)\)

\begin{Shaded}
\begin{Highlighting}[]
\NormalTok{g\_L }\OtherTok{\textless{}{-}} \ControlFlowTok{function}\NormalTok{(L) \{}
  \FunctionTok{return}\NormalTok{(}\FunctionTok{dnorm}\NormalTok{(L, }\AttributeTok{mean =} \DecValTok{3}\NormalTok{, }\AttributeTok{sd =} \DecValTok{2}\NormalTok{))}
\NormalTok{\}}
\end{Highlighting}
\end{Shaded}

Una vez creadas las funciones anteriores, se procede a crear el
algoritmo de muestreo por importancia.

\begin{Shaded}
\begin{Highlighting}[]
\CommentTok{\# Establecemos la semilla}

\NormalTok{n }\OtherTok{\textless{}{-}} \DecValTok{10}\SpecialCharTok{\^{}}\DecValTok{4}  \CommentTok{\# número de muestras}

\CommentTok{\# Generar muestras de la distribución normal g}
\NormalTok{L\_samples }\OtherTok{\textless{}{-}} \FunctionTok{rnorm}\NormalTok{(n, }\AttributeTok{mean =} \DecValTok{3}\NormalTok{, }\AttributeTok{sd =} \DecValTok{2}\NormalTok{)}

\CommentTok{\# Estimamos el valor esperado usando la importancia ponderada}
\NormalTok{pesos }\OtherTok{\textless{}{-}} \FunctionTok{f\_L}\NormalTok{(L\_samples) }\SpecialCharTok{/} \FunctionTok{g\_L}\NormalTok{(L\_samples)}
\NormalTok{perdidas\_esperadas }\OtherTok{\textless{}{-}} \FunctionTok{mean}\NormalTok{(L\_samples }\SpecialCharTok{*}\NormalTok{ pesos)}

\FunctionTok{cat}\NormalTok{(}\StringTok{"La estimación del valor esperado de la pérdida usando muestreo por importancia es:"}\NormalTok{, }
\NormalTok{    perdidas\_esperadas, }\StringTok{"}\SpecialCharTok{\textbackslash{}n}\StringTok{"}\NormalTok{)}
\end{Highlighting}
\end{Shaded}

\begin{verbatim}
## La estimación del valor esperado de la pérdida usando muestreo por importancia es: -553.1743
\end{verbatim}

\section{Ejercicio 3}\label{ejercicio-3}

Se definen los tiempos entre accidentes y las funciones exponencial y
gamma, ademas se establecen los parametros de la funcion gamma, todo
esto segun lo establecido en el enunciado.

\begin{Shaded}
\begin{Highlighting}[]
\NormalTok{tiempos }\OtherTok{\textless{}{-}} \FunctionTok{c}\NormalTok{(}\FloatTok{2.72}\NormalTok{, }\FloatTok{1.93}\NormalTok{, }\FloatTok{1.76}\NormalTok{, }\FloatTok{0.49}\NormalTok{, }\FloatTok{6.12}\NormalTok{, }\FloatTok{0.43}\NormalTok{, }\FloatTok{4.01}\NormalTok{, }\FloatTok{1.71}\NormalTok{, }\FloatTok{2.01}\NormalTok{, }\FloatTok{5.96}\NormalTok{)}

\NormalTok{lambda\_est }\OtherTok{\textless{}{-}} \DecValTok{1} \SpecialCharTok{/} \FunctionTok{mean}\NormalTok{(tiempos)}

\CommentTok{\# Máximo teórico de la densidad aux (gamma)}
\NormalTok{m }\OtherTok{\textless{}{-}} \FunctionTok{max}\NormalTok{(}\FunctionTok{dgamma}\NormalTok{(}\DecValTok{1}\NormalTok{, }\DecValTok{2}\NormalTok{, }\DecValTok{1}\NormalTok{))}

\CommentTok{\# Simulaciones}
\NormalTok{U }\OtherTok{\textless{}{-}} \FunctionTok{runif}\NormalTok{(}\DecValTok{10}\SpecialCharTok{\^{}}\DecValTok{4}\NormalTok{)}
\NormalTok{x }\OtherTok{\textless{}{-}} \FunctionTok{rgamma}\NormalTok{(}\DecValTok{10}\SpecialCharTok{\^{}}\DecValTok{4}\NormalTok{, }\DecValTok{2}\NormalTok{, }\DecValTok{1}\NormalTok{)  }\CommentTok{\# Muestras de la dist aux (gamma)}
\NormalTok{ngen }\OtherTok{\textless{}{-}} \FunctionTok{length}\NormalTok{(x)}

\CommentTok{\# Dist exponecial (objetivo)}
\NormalTok{dexp1 }\OtherTok{\textless{}{-}} \FunctionTok{Vectorize}\NormalTok{(}\ControlFlowTok{function}\NormalTok{(x) }\FunctionTok{dexp}\NormalTok{(x, lambda\_est))}
\end{Highlighting}
\end{Shaded}

Algoritmo de aceptacion-rechazo

\begin{Shaded}
\begin{Highlighting}[]
\ControlFlowTok{for}\NormalTok{(i }\ControlFlowTok{in} \DecValTok{1}\SpecialCharTok{:}\DecValTok{10}\SpecialCharTok{\^{}}\DecValTok{4}\NormalTok{)\{}
  \ControlFlowTok{while}\NormalTok{((U[i] }\SpecialCharTok{*}\NormalTok{ m) }\SpecialCharTok{\textgreater{}=} \FunctionTok{dexp1}\NormalTok{(x[i]))\{ }
\NormalTok{    U[i] }\OtherTok{\textless{}{-}} \FunctionTok{runif}\NormalTok{(}\DecValTok{1}\NormalTok{)}
\NormalTok{    x[i] }\OtherTok{\textless{}{-}} \FunctionTok{rgamma}\NormalTok{(}\DecValTok{1}\NormalTok{, }\DecValTok{2}\NormalTok{, }\DecValTok{1}\NormalTok{)}
\NormalTok{    ngen }\OtherTok{\textless{}{-}}\NormalTok{ ngen }\SpecialCharTok{+} \DecValTok{1}
\NormalTok{  \}}
\NormalTok{\}}
\end{Highlighting}
\end{Shaded}

Resultados

\begin{Shaded}
\begin{Highlighting}[]
\FunctionTok{cat}\NormalTok{(}\StringTok{"Número de generaciones = "}\NormalTok{, ngen, }\StringTok{"}\SpecialCharTok{\textbackslash{}n}\StringTok{"}\NormalTok{)}
\end{Highlighting}
\end{Shaded}

\begin{verbatim}
## Número de generaciones =  18718
\end{verbatim}

\begin{Shaded}
\begin{Highlighting}[]
\FunctionTok{cat}\NormalTok{(}\StringTok{"Número medio de aceptados = "}\NormalTok{, ngen }\SpecialCharTok{/} \DecValTok{10}\SpecialCharTok{\^{}}\DecValTok{4}\NormalTok{, }\StringTok{"}\SpecialCharTok{\textbackslash{}n}\StringTok{"}\NormalTok{)}
\end{Highlighting}
\end{Shaded}

\begin{verbatim}
## Número medio de aceptados =  1.8718
\end{verbatim}

\begin{Shaded}
\begin{Highlighting}[]
\FunctionTok{cat}\NormalTok{(}\StringTok{"Proporción de rechazos = "}\NormalTok{, }\DecValTok{1} \SpecialCharTok{{-}} \DecValTok{10}\SpecialCharTok{\^{}}\DecValTok{4} \SpecialCharTok{/}\NormalTok{ ngen, }\StringTok{"}\SpecialCharTok{\textbackslash{}n}\StringTok{"}\NormalTok{)}
\end{Highlighting}
\end{Shaded}

\begin{verbatim}
## Proporción de rechazos =  0.4657549
\end{verbatim}

\begin{Shaded}
\begin{Highlighting}[]
\FunctionTok{hist}\NormalTok{(x, }\AttributeTok{breaks =} \StringTok{"FD"}\NormalTok{, }\AttributeTok{freq =} \ConstantTok{FALSE}\NormalTok{, }\AttributeTok{main =} \StringTok{"Histograma de \textbackslash{}u03BB estimados"}\NormalTok{)}
\FunctionTok{curve}\NormalTok{(}\FunctionTok{dexp}\NormalTok{(x, lambda\_est), }\AttributeTok{col =} \DecValTok{2}\NormalTok{, }\AttributeTok{lwd =} \DecValTok{2}\NormalTok{, }\AttributeTok{add =} \ConstantTok{TRUE}\NormalTok{)}
\end{Highlighting}
\end{Shaded}

\includegraphics{Tarea2_EstadisticaII_files/figure-latex/unnamed-chunk-11-1.pdf}

Intervalo de credibilidad al 99\%

\begin{Shaded}
\begin{Highlighting}[]
\NormalTok{cred\_interval }\OtherTok{\textless{}{-}} \FunctionTok{quantile}\NormalTok{(x, }\AttributeTok{probs =} \FunctionTok{c}\NormalTok{(}\FloatTok{0.005}\NormalTok{, }\FloatTok{0.995}\NormalTok{))}
\FunctionTok{cat}\NormalTok{(}\StringTok{"Intervalo de credibilidad al 99\%: ["}\NormalTok{, cred\_interval[}\DecValTok{1}\NormalTok{], }\StringTok{", "}\NormalTok{, cred\_interval[}\DecValTok{2}\NormalTok{], }\StringTok{"]}\SpecialCharTok{\textbackslash{}n}\StringTok{"}\NormalTok{)}
\end{Highlighting}
\end{Shaded}

\begin{verbatim}
## Intervalo de credibilidad al 99%: [ 0.07527111 ,  5.484737 ]
\end{verbatim}

Aceptacion o rechazo de lambda = 5

\begin{Shaded}
\begin{Highlighting}[]
\NormalTok{lambda\_hip }\OtherTok{\textless{}{-}} \FloatTok{0.5}
\ControlFlowTok{if}\NormalTok{(lambda\_hip }\SpecialCharTok{\textgreater{}=}\NormalTok{ cred\_interval[}\DecValTok{1}\NormalTok{] }\SpecialCharTok{\&\&}\NormalTok{ lambda\_hip }\SpecialCharTok{\textless{}=}\NormalTok{ cred\_interval[}\DecValTok{2}\NormalTok{]) \{}
  \FunctionTok{cat}\NormalTok{(}\StringTok{"No se rechaza la hipótesis lambda = 0.5, está dentro del intervalo de credibilidad.}\SpecialCharTok{\textbackslash{}n}\StringTok{"}\NormalTok{)}
\NormalTok{\} }\ControlFlowTok{else}\NormalTok{ \{}
  \FunctionTok{cat}\NormalTok{(}\StringTok{"Se rechaza la hipótesis lambda = 0.5, está fuera del intervalo de credibilidad.}\SpecialCharTok{\textbackslash{}n}\StringTok{"}\NormalTok{)}
\NormalTok{\}}
\end{Highlighting}
\end{Shaded}

\begin{verbatim}
## No se rechaza la hipótesis lambda = 0.5, está dentro del intervalo de credibilidad.
\end{verbatim}

\section{Ejercicio 4}\label{ejercicio-4}

Funcion \[
f(x) = exp(\frac{sen(10x)}{10cos(x)})
\]

\begin{Shaded}
\begin{Highlighting}[]
\NormalTok{f }\OtherTok{\textless{}{-}} \ControlFlowTok{function}\NormalTok{(x) \{}
  \FunctionTok{return}\NormalTok{(}\FunctionTok{exp}\NormalTok{(}\FunctionTok{sin}\NormalTok{(}\DecValTok{10} \SpecialCharTok{*}\NormalTok{ x) }\SpecialCharTok{/}\NormalTok{ (}\DecValTok{10} \SpecialCharTok{*} \FunctionTok{cos}\NormalTok{(x))))}
\NormalTok{\}}
\end{Highlighting}
\end{Shaded}

Funcion de recalentamiento simulado

\begin{Shaded}
\begin{Highlighting}[]
\NormalTok{recalentamientoSimulado }\OtherTok{\textless{}{-}} \ControlFlowTok{function}\NormalTok{(f, x0, T0, alpha, iterMax) \{}
\NormalTok{  xActual }\OtherTok{\textless{}{-}}\NormalTok{ x0}
\NormalTok{  fActual }\OtherTok{\textless{}{-}} \FunctionTok{f}\NormalTok{(xActual)}
\NormalTok{  T }\OtherTok{\textless{}{-}}\NormalTok{ T0}
\NormalTok{  evolucion }\OtherTok{\textless{}{-}} \FunctionTok{numeric}\NormalTok{(iterMax)}
  
  \ControlFlowTok{for}\NormalTok{ (i }\ControlFlowTok{in} \DecValTok{1}\SpecialCharTok{:}\NormalTok{iterMax) \{}
\NormalTok{    evolucion[i] }\OtherTok{\textless{}{-}}\NormalTok{ xActual}
    \CommentTok{\# Generar un vecino aleatorio en el vecindario de xActual}
\NormalTok{    xNuevo }\OtherTok{\textless{}{-}}\NormalTok{ xActual }\SpecialCharTok{+} \FunctionTok{runif}\NormalTok{(}\DecValTok{1}\NormalTok{, }\SpecialCharTok{{-}}\DecValTok{1}\NormalTok{, }\DecValTok{1}\NormalTok{)}
    \ControlFlowTok{if}\NormalTok{ (xNuevo }\SpecialCharTok{\textless{}} \DecValTok{0} \SpecialCharTok{||}\NormalTok{ xNuevo }\SpecialCharTok{\textgreater{}} \DecValTok{10}\NormalTok{) \{}
      \ControlFlowTok{next}  \CommentTok{\# Saltar si xNuevo está fuera del intervalo [0, 10]}
\NormalTok{    \}}
    
    \CommentTok{\# Evaluar la nueva solución}
\NormalTok{    fNuevo }\OtherTok{\textless{}{-}} \FunctionTok{f}\NormalTok{(xNuevo)}
    \ControlFlowTok{if}\NormalTok{ (fNuevo }\SpecialCharTok{\textless{}}\NormalTok{ fActual }\SpecialCharTok{||} \FunctionTok{runif}\NormalTok{(}\DecValTok{1}\NormalTok{) }\SpecialCharTok{\textless{}} \FunctionTok{exp}\NormalTok{((fActual }\SpecialCharTok{{-}}\NormalTok{ fNuevo) }\SpecialCharTok{/}\NormalTok{ T)) \{}
\NormalTok{      xActual }\OtherTok{\textless{}{-}}\NormalTok{ xNuevo}
\NormalTok{      fActual }\OtherTok{\textless{}{-}}\NormalTok{ fNuevo}
\NormalTok{    \}}
    
\NormalTok{    T }\OtherTok{\textless{}{-}}\NormalTok{ alpha }\SpecialCharTok{*}\NormalTok{ T}
\NormalTok{  \}}
  
  \FunctionTok{return}\NormalTok{(}\FunctionTok{list}\NormalTok{(}\AttributeTok{minimo =}\NormalTok{ xActual, }\AttributeTok{historial =}\NormalTok{ evolucion))}
\NormalTok{\}}
\end{Highlighting}
\end{Shaded}

Parametros y aplicacion

\begin{Shaded}
\begin{Highlighting}[]
\NormalTok{x0 }\OtherTok{\textless{}{-}} \DecValTok{5}
\NormalTok{T0 }\OtherTok{\textless{}{-}} \DecValTok{1}
\NormalTok{alpha }\OtherTok{\textless{}{-}} \FloatTok{0.99}
\NormalTok{iterMax }\OtherTok{\textless{}{-}} \DecValTok{1000}

\NormalTok{resultado }\OtherTok{\textless{}{-}} \FunctionTok{recalentamientoSimulado}\NormalTok{(f, x0, T0, alpha, iterMax)}
\end{Highlighting}
\end{Shaded}

Resultados

\begin{Shaded}
\begin{Highlighting}[]
\FunctionTok{cat}\NormalTok{(}\StringTok{"El valor mínimo estimado de f(x) es en x ="}\NormalTok{, resultado}\SpecialCharTok{$}\NormalTok{minimo, }\StringTok{"}\SpecialCharTok{\textbackslash{}n}\StringTok{"}\NormalTok{)}
\end{Highlighting}
\end{Shaded}

\begin{verbatim}
## El valor mínimo estimado de f(x) es en x = 8.301542
\end{verbatim}

\begin{Shaded}
\begin{Highlighting}[]
\CommentTok{\# Evolucion de los estados}
\FunctionTok{plot}\NormalTok{(resultado}\SpecialCharTok{$}\NormalTok{historial, }\AttributeTok{type =} \StringTok{"l"}\NormalTok{, }\AttributeTok{col =} \StringTok{"blue"}\NormalTok{, }
     \AttributeTok{main =} \StringTok{"Estados de la cadena en el Recalentamiento Simulado"}\NormalTok{,}
     \AttributeTok{xlab =} \StringTok{"Iteraciones"}\NormalTok{, }\AttributeTok{ylab =} \StringTok{"Valor de x"}\NormalTok{)}
\end{Highlighting}
\end{Shaded}

\includegraphics{Tarea2_EstadisticaII_files/figure-latex/unnamed-chunk-17-1.pdf}

\end{document}
