% Options for packages loaded elsewhere
\PassOptionsToPackage{unicode}{hyperref}
\PassOptionsToPackage{hyphens}{url}
%
\documentclass[
]{article}
\usepackage{amsmath,amssymb}
\usepackage{iftex}
\ifPDFTeX
  \usepackage[T1]{fontenc}
  \usepackage[utf8]{inputenc}
  \usepackage{textcomp} % provide euro and other symbols
\else % if luatex or xetex
  \usepackage{unicode-math} % this also loads fontspec
  \defaultfontfeatures{Scale=MatchLowercase}
  \defaultfontfeatures[\rmfamily]{Ligatures=TeX,Scale=1}
\fi
\usepackage{lmodern}
\ifPDFTeX\else
  % xetex/luatex font selection
\fi
% Use upquote if available, for straight quotes in verbatim environments
\IfFileExists{upquote.sty}{\usepackage{upquote}}{}
\IfFileExists{microtype.sty}{% use microtype if available
  \usepackage[]{microtype}
  \UseMicrotypeSet[protrusion]{basicmath} % disable protrusion for tt fonts
}{}
\makeatletter
\@ifundefined{KOMAClassName}{% if non-KOMA class
  \IfFileExists{parskip.sty}{%
    \usepackage{parskip}
  }{% else
    \setlength{\parindent}{0pt}
    \setlength{\parskip}{6pt plus 2pt minus 1pt}}
}{% if KOMA class
  \KOMAoptions{parskip=half}}
\makeatother
\usepackage{xcolor}
\usepackage[margin=1in]{geometry}
\usepackage{color}
\usepackage{fancyvrb}
\newcommand{\VerbBar}{|}
\newcommand{\VERB}{\Verb[commandchars=\\\{\}]}
\DefineVerbatimEnvironment{Highlighting}{Verbatim}{commandchars=\\\{\}}
% Add ',fontsize=\small' for more characters per line
\usepackage{framed}
\definecolor{shadecolor}{RGB}{248,248,248}
\newenvironment{Shaded}{\begin{snugshade}}{\end{snugshade}}
\newcommand{\AlertTok}[1]{\textcolor[rgb]{0.94,0.16,0.16}{#1}}
\newcommand{\AnnotationTok}[1]{\textcolor[rgb]{0.56,0.35,0.01}{\textbf{\textit{#1}}}}
\newcommand{\AttributeTok}[1]{\textcolor[rgb]{0.13,0.29,0.53}{#1}}
\newcommand{\BaseNTok}[1]{\textcolor[rgb]{0.00,0.00,0.81}{#1}}
\newcommand{\BuiltInTok}[1]{#1}
\newcommand{\CharTok}[1]{\textcolor[rgb]{0.31,0.60,0.02}{#1}}
\newcommand{\CommentTok}[1]{\textcolor[rgb]{0.56,0.35,0.01}{\textit{#1}}}
\newcommand{\CommentVarTok}[1]{\textcolor[rgb]{0.56,0.35,0.01}{\textbf{\textit{#1}}}}
\newcommand{\ConstantTok}[1]{\textcolor[rgb]{0.56,0.35,0.01}{#1}}
\newcommand{\ControlFlowTok}[1]{\textcolor[rgb]{0.13,0.29,0.53}{\textbf{#1}}}
\newcommand{\DataTypeTok}[1]{\textcolor[rgb]{0.13,0.29,0.53}{#1}}
\newcommand{\DecValTok}[1]{\textcolor[rgb]{0.00,0.00,0.81}{#1}}
\newcommand{\DocumentationTok}[1]{\textcolor[rgb]{0.56,0.35,0.01}{\textbf{\textit{#1}}}}
\newcommand{\ErrorTok}[1]{\textcolor[rgb]{0.64,0.00,0.00}{\textbf{#1}}}
\newcommand{\ExtensionTok}[1]{#1}
\newcommand{\FloatTok}[1]{\textcolor[rgb]{0.00,0.00,0.81}{#1}}
\newcommand{\FunctionTok}[1]{\textcolor[rgb]{0.13,0.29,0.53}{\textbf{#1}}}
\newcommand{\ImportTok}[1]{#1}
\newcommand{\InformationTok}[1]{\textcolor[rgb]{0.56,0.35,0.01}{\textbf{\textit{#1}}}}
\newcommand{\KeywordTok}[1]{\textcolor[rgb]{0.13,0.29,0.53}{\textbf{#1}}}
\newcommand{\NormalTok}[1]{#1}
\newcommand{\OperatorTok}[1]{\textcolor[rgb]{0.81,0.36,0.00}{\textbf{#1}}}
\newcommand{\OtherTok}[1]{\textcolor[rgb]{0.56,0.35,0.01}{#1}}
\newcommand{\PreprocessorTok}[1]{\textcolor[rgb]{0.56,0.35,0.01}{\textit{#1}}}
\newcommand{\RegionMarkerTok}[1]{#1}
\newcommand{\SpecialCharTok}[1]{\textcolor[rgb]{0.81,0.36,0.00}{\textbf{#1}}}
\newcommand{\SpecialStringTok}[1]{\textcolor[rgb]{0.31,0.60,0.02}{#1}}
\newcommand{\StringTok}[1]{\textcolor[rgb]{0.31,0.60,0.02}{#1}}
\newcommand{\VariableTok}[1]{\textcolor[rgb]{0.00,0.00,0.00}{#1}}
\newcommand{\VerbatimStringTok}[1]{\textcolor[rgb]{0.31,0.60,0.02}{#1}}
\newcommand{\WarningTok}[1]{\textcolor[rgb]{0.56,0.35,0.01}{\textbf{\textit{#1}}}}
\usepackage{graphicx}
\makeatletter
\def\maxwidth{\ifdim\Gin@nat@width>\linewidth\linewidth\else\Gin@nat@width\fi}
\def\maxheight{\ifdim\Gin@nat@height>\textheight\textheight\else\Gin@nat@height\fi}
\makeatother
% Scale images if necessary, so that they will not overflow the page
% margins by default, and it is still possible to overwrite the defaults
% using explicit options in \includegraphics[width, height, ...]{}
\setkeys{Gin}{width=\maxwidth,height=\maxheight,keepaspectratio}
% Set default figure placement to htbp
\makeatletter
\def\fps@figure{htbp}
\makeatother
\setlength{\emergencystretch}{3em} % prevent overfull lines
\providecommand{\tightlist}{%
  \setlength{\itemsep}{0pt}\setlength{\parskip}{0pt}}
\setcounter{secnumdepth}{-\maxdimen} % remove section numbering
\ifLuaTeX
  \usepackage{selnolig}  % disable illegal ligatures
\fi
\IfFileExists{bookmark.sty}{\usepackage{bookmark}}{\usepackage{hyperref}}
\IfFileExists{xurl.sty}{\usepackage{xurl}}{} % add URL line breaks if available
\urlstyle{same}
\hypersetup{
  pdftitle={Tarea 2 - CA0307: Estadística Actuarial II},
  pdfauthor={Anthony Mauricio Jiménez Navarro \textbar{} C24067; Henri Gerard Gabert Hidalgo \textbar{} B93096; Juan Pablo Morgan Sandí \textbar{} C15319},
  hidelinks,
  pdfcreator={LaTeX via pandoc}}

\title{Tarea 2 - CA0307: Estadística Actuarial II}
\author{Anthony Mauricio Jiménez Navarro \textbar{} C24067 \and Henri
Gerard Gabert Hidalgo \textbar{} B93096 \and Juan Pablo Morgan Sandí
\textbar{} C15319}
\date{2024-10-22}

\begin{document}
\maketitle

{
\setcounter{tocdepth}{2}
\tableofcontents
}
\hypertarget{librerias}{%
\section{Librerias}\label{librerias}}

\begin{Shaded}
\begin{Highlighting}[]
\CommentTok{\#set.seed(2024)}
\end{Highlighting}
\end{Shaded}

\hypertarget{ejercicio-1}{%
\section{Ejercicio 1}\label{ejercicio-1}}

Primero se crea la función a integrar, la cual sabemos que es \[
\int_0^1 \frac{e^{-x^2}}{1+x^2} d x
\]

\begin{Shaded}
\begin{Highlighting}[]
\NormalTok{f }\OtherTok{\textless{}{-}} \ControlFlowTok{function}\NormalTok{(x) \{}
  \FunctionTok{exp}\NormalTok{(}\SpecialCharTok{{-}}\NormalTok{x}\SpecialCharTok{\^{}}\DecValTok{2}\NormalTok{) }\SpecialCharTok{/}\NormalTok{ (}\DecValTok{1} \SpecialCharTok{+}\NormalTok{ x}\SpecialCharTok{\^{}}\DecValTok{2}\NormalTok{)}
\NormalTok{\}}
\end{Highlighting}
\end{Shaded}

Una vez hecho lo anterior, programos el algoritmo de Montecarlo.

\begin{Shaded}
\begin{Highlighting}[]
\CommentTok{\# Método de Montecarlo para aproximar la integral}
\NormalTok{montecarlo\_integration }\OtherTok{\textless{}{-}} \ControlFlowTok{function}\NormalTok{(N) \{}
  \CommentTok{\# Generamos N muestras aleatorias entre 0 y 1}
\NormalTok{  x }\OtherTok{\textless{}{-}} \FunctionTok{runif}\NormalTok{(N, }\DecValTok{0}\NormalTok{, }\DecValTok{1}\NormalTok{)}
  
  \CommentTok{\# Evaluamos la función en los puntos muestreados}
\NormalTok{  fx }\OtherTok{\textless{}{-}} \FunctionTok{f}\NormalTok{(x)}
  
  \CommentTok{\# Estimar la integral como el promedio de f(x)}
\NormalTok{  integral\_estimate }\OtherTok{\textless{}{-}} \FunctionTok{mean}\NormalTok{(fx)}

  \FunctionTok{return}\NormalTok{(integral\_estimate)}
\NormalTok{\}}

\NormalTok{N }\OtherTok{\textless{}{-}} \DecValTok{100000}  \CommentTok{\# número de muestras}

\NormalTok{montecarlo\_result }\OtherTok{\textless{}{-}} \FunctionTok{montecarlo\_integration}\NormalTok{(N)}

\FunctionTok{cat}\NormalTok{(}\StringTok{"Aproximación de la integral por Montecarlo:"}\NormalTok{, montecarlo\_result, }\StringTok{"}\SpecialCharTok{\textbackslash{}n}\StringTok{"}\NormalTok{)}
\end{Highlighting}
\end{Shaded}

\begin{verbatim}
## Aproximación de la integral por Montecarlo: 0.6189672
\end{verbatim}

Ahora, usando integrate.

\begin{Shaded}
\begin{Highlighting}[]
\NormalTok{integral\_exacta }\OtherTok{\textless{}{-}} \FunctionTok{integrate}\NormalTok{(f, }\DecValTok{0}\NormalTok{, }\DecValTok{1}\NormalTok{)}
\FunctionTok{cat}\NormalTok{(}\StringTok{"Aproximación de la integral por Integrate:"}\NormalTok{, integral\_exacta}\SpecialCharTok{$}\NormalTok{value, }\StringTok{"}\SpecialCharTok{\textbackslash{}n}\StringTok{"}\NormalTok{)}
\end{Highlighting}
\end{Shaded}

\begin{verbatim}
## Aproximación de la integral por Integrate: 0.618822
\end{verbatim}

\begin{Shaded}
\begin{Highlighting}[]
\FunctionTok{cat}\NormalTok{(}\StringTok{"El error absoluto de Integrate:"}\NormalTok{, integral\_exacta}\SpecialCharTok{$}\NormalTok{abs.error, }\StringTok{"}\SpecialCharTok{\textbackslash{}n}\StringTok{"}\NormalTok{)}
\end{Highlighting}
\end{Shaded}

\begin{verbatim}
## El error absoluto de Integrate: 6.870304e-15
\end{verbatim}

Ahora, la diferencia entre el resultado de Montecarlo y el de Integrate
es:

\begin{Shaded}
\begin{Highlighting}[]
\NormalTok{montecarlo\_result }\SpecialCharTok{{-}}\NormalTok{ integral\_exacta}\SpecialCharTok{$}\NormalTok{value}
\end{Highlighting}
\end{Shaded}

\begin{verbatim}
## [1] 0.0001451988
\end{verbatim}

\hypertarget{ejercicio-2}{%
\section{Ejercicio 2}\label{ejercicio-2}}

Primero, se crea la función \(f_L\), la cual es \[
f_L(L)= \lambda e^{-\lambda L}
\] La cual, sabiendo que \(\lambda = 1\), es \(f_L(L)= e^{-L}\)

\begin{Shaded}
\begin{Highlighting}[]
\NormalTok{f\_L }\OtherTok{\textless{}{-}} \ControlFlowTok{function}\NormalTok{(L) \{}
  \FunctionTok{return}\NormalTok{(}\FunctionTok{exp}\NormalTok{(}\SpecialCharTok{{-}}\NormalTok{L)) }
\NormalTok{\}}
\end{Highlighting}
\end{Shaded}

Por el enunciado sabemos también que \(g(L) \sim N(3, 4)\)

\begin{Shaded}
\begin{Highlighting}[]
\NormalTok{g\_L }\OtherTok{\textless{}{-}} \ControlFlowTok{function}\NormalTok{(L) \{}
  \FunctionTok{return}\NormalTok{(}\FunctionTok{dnorm}\NormalTok{(L, }\AttributeTok{mean =} \DecValTok{3}\NormalTok{, }\AttributeTok{sd =} \DecValTok{2}\NormalTok{))}
\NormalTok{\}}
\end{Highlighting}
\end{Shaded}

Una vez creadas las funciones anteriores, se procede a crear el
algoritmo de muestreo por importancia.

\begin{Shaded}
\begin{Highlighting}[]
\CommentTok{\# Establecemos la semilla}
\FunctionTok{set.seed}\NormalTok{(}\DecValTok{54321}\NormalTok{)}

\NormalTok{n }\OtherTok{\textless{}{-}} \DecValTok{10}\SpecialCharTok{\^{}}\DecValTok{4}  \CommentTok{\# número de muestras}
\NormalTok{L\_samples }\OtherTok{\textless{}{-}} \FunctionTok{numeric}\NormalTok{(}\DecValTok{0}\NormalTok{)  }\CommentTok{\# Inicializamos el vector de muestras}

\CommentTok{\# Rechazamos valores negativos}
\ControlFlowTok{while}\NormalTok{ (}\FunctionTok{length}\NormalTok{(L\_samples) }\SpecialCharTok{\textless{}}\NormalTok{ n) \{}
\NormalTok{  samples }\OtherTok{\textless{}{-}} \FunctionTok{rnorm}\NormalTok{(n, }\AttributeTok{mean =} \DecValTok{3}\NormalTok{, }\AttributeTok{sd =} \DecValTok{2}\NormalTok{)  }\CommentTok{\# Generamos n muestras de N(3, 2\^{}2)}
\NormalTok{  L\_samples }\OtherTok{\textless{}{-}} \FunctionTok{c}\NormalTok{(L\_samples, samples[samples }\SpecialCharTok{\textgreater{}} \DecValTok{0}\NormalTok{])  }\CommentTok{\# Solo conservamos las positivas por la restricción}
\NormalTok{  L\_samples }\OtherTok{\textless{}{-}}\NormalTok{ L\_samples[}\DecValTok{1}\SpecialCharTok{:}\NormalTok{n]  }\CommentTok{\# Aseguramos que solo tengamos n muestras}
\NormalTok{\}}

\NormalTok{pesos }\OtherTok{\textless{}{-}} \FunctionTok{f\_L}\NormalTok{(L\_samples) }\SpecialCharTok{/} \FunctionTok{g\_L}\NormalTok{(L\_samples)}
\CommentTok{\# Filtrar valores infinitos o NA en los pesos de importancia}
\NormalTok{pesos\_validos }\OtherTok{\textless{}{-}} \FunctionTok{is.finite}\NormalTok{(pesos) }\SpecialCharTok{\&} \SpecialCharTok{!}\FunctionTok{is.na}\NormalTok{(pesos)}
\NormalTok{L\_samples }\OtherTok{\textless{}{-}}\NormalTok{ L\_samples[pesos\_validos]}
\NormalTok{pesos }\OtherTok{\textless{}{-}}\NormalTok{ pesos[pesos\_validos]}

\CommentTok{\# Calcular la pérdida esperada}
\NormalTok{perdidas\_esperadas }\OtherTok{\textless{}{-}} \FunctionTok{mean}\NormalTok{(L\_samples }\SpecialCharTok{*}\NormalTok{ pesos)}

\FunctionTok{cat}\NormalTok{(}\StringTok{"La estimación del valor esperado de la pérdida usando muestreo por importancia es:"}\NormalTok{, }
\NormalTok{    perdidas\_esperadas, }\StringTok{"}\SpecialCharTok{\textbackslash{}n}\StringTok{"}\NormalTok{)}
\end{Highlighting}
\end{Shaded}

\begin{verbatim}
## La estimación del valor esperado de la pérdida usando muestreo por importancia es: 1.069952
\end{verbatim}

\hypertarget{ejercicio-3}{%
\section{Ejercicio 3}\label{ejercicio-3}}

Se definen los tiempos entre accidentes y las funciones exponencial y
gamma, ademas se establecen los parametros de la funcion gamma, todo
esto segun lo establecido en el enunciado.

\begin{Shaded}
\begin{Highlighting}[]
\NormalTok{tiempos }\OtherTok{\textless{}{-}} \FunctionTok{c}\NormalTok{(}\FloatTok{2.72}\NormalTok{, }\FloatTok{1.93}\NormalTok{, }\FloatTok{1.76}\NormalTok{, }\FloatTok{0.49}\NormalTok{, }\FloatTok{6.12}\NormalTok{, }\FloatTok{0.43}\NormalTok{, }\FloatTok{4.01}\NormalTok{, }\FloatTok{1.71}\NormalTok{, }\FloatTok{2.01}\NormalTok{, }\FloatTok{5.96}\NormalTok{)}

\CommentTok{\# Definición de las funciones de densidad}
\NormalTok{objetivo }\OtherTok{\textless{}{-}} \ControlFlowTok{function}\NormalTok{(lambda, x) \{}
\NormalTok{  lambda }\SpecialCharTok{*} \FunctionTok{exp}\NormalTok{(}\SpecialCharTok{{-}}\NormalTok{lambda }\SpecialCharTok{*}\NormalTok{ x)}
\NormalTok{\}}

\NormalTok{previa }\OtherTok{\textless{}{-}} \ControlFlowTok{function}\NormalTok{(lambda) \{}
  \FunctionTok{dgamma}\NormalTok{(lambda, }\AttributeTok{shape =} \DecValTok{2}\NormalTok{, }\AttributeTok{rate =} \DecValTok{1}\NormalTok{)}
\NormalTok{\}}
\end{Highlighting}
\end{Shaded}

Algoritmo de aceptacion-rechazo

\begin{Shaded}
\begin{Highlighting}[]
\NormalTok{n }\OtherTok{\textless{}{-}} \DecValTok{10000}
\NormalTok{valoresLambda }\OtherTok{\textless{}{-}} \FunctionTok{numeric}\NormalTok{(n)}

\ControlFlowTok{for}\NormalTok{ (i }\ControlFlowTok{in} \DecValTok{1}\SpecialCharTok{:}\NormalTok{n) \{}
  \ControlFlowTok{repeat}\NormalTok{ \{}
    \CommentTok{\# Generar candidato de la distribución prior}
\NormalTok{    lambda }\OtherTok{\textless{}{-}} \FunctionTok{rgamma}\NormalTok{(}\DecValTok{1}\NormalTok{, }\AttributeTok{shape =} \DecValTok{2}\NormalTok{, }\AttributeTok{rate =} \DecValTok{1}\NormalTok{)}
    
    \CommentTok{\# Generar variable uniforme}
\NormalTok{    u }\OtherTok{\textless{}{-}} \FunctionTok{runif}\NormalTok{(}\DecValTok{1}\NormalTok{)}
    
    \CommentTok{\# Verificar aceptación}
    \ControlFlowTok{if}\NormalTok{ (u }\SpecialCharTok{\textless{}} \FunctionTok{min}\NormalTok{(}\FunctionTok{objetivo}\NormalTok{(lambda, tiempos) }\SpecialCharTok{/} \FunctionTok{previa}\NormalTok{(lambda))) \{}
\NormalTok{      valoresLambda[i] }\OtherTok{\textless{}{-}}\NormalTok{ lambda}
      \ControlFlowTok{break}
\NormalTok{    \}}
\NormalTok{  \}}
\NormalTok{\}}
\end{Highlighting}
\end{Shaded}

Resultados

\begin{Shaded}
\begin{Highlighting}[]
\NormalTok{ngen }\OtherTok{\textless{}{-}}\NormalTok{ n  }
\FunctionTok{cat}\NormalTok{(}\StringTok{"Número de generaciones = "}\NormalTok{, ngen, }\StringTok{"}\SpecialCharTok{\textbackslash{}n}\StringTok{"}\NormalTok{)}
\end{Highlighting}
\end{Shaded}

\begin{verbatim}
## Número de generaciones =  10000
\end{verbatim}

\begin{Shaded}
\begin{Highlighting}[]
\FunctionTok{cat}\NormalTok{(}\StringTok{"Número medio de aceptados = "}\NormalTok{, ngen }\SpecialCharTok{/} \DecValTok{10}\SpecialCharTok{\^{}}\DecValTok{4}\NormalTok{, }\StringTok{"}\SpecialCharTok{\textbackslash{}n}\StringTok{"}\NormalTok{)}
\end{Highlighting}
\end{Shaded}

\begin{verbatim}
## Número medio de aceptados =  1
\end{verbatim}

\begin{Shaded}
\begin{Highlighting}[]
\FunctionTok{cat}\NormalTok{(}\StringTok{"Proporción de rechazos = "}\NormalTok{, }\DecValTok{1} \SpecialCharTok{{-}} \DecValTok{10}\SpecialCharTok{\^{}}\DecValTok{4} \SpecialCharTok{/}\NormalTok{ ngen, }\StringTok{"}\SpecialCharTok{\textbackslash{}n}\StringTok{"}\NormalTok{)}
\end{Highlighting}
\end{Shaded}

\begin{verbatim}
## Proporción de rechazos =  0
\end{verbatim}

\begin{Shaded}
\begin{Highlighting}[]
\NormalTok{lambda\_est }\OtherTok{\textless{}{-}} \DecValTok{1} \SpecialCharTok{/} \FunctionTok{mean}\NormalTok{(tiempos)}

\FunctionTok{hist}\NormalTok{(valoresLambda, }\AttributeTok{breaks =} \StringTok{"FD"}\NormalTok{, }\AttributeTok{freq =} \ConstantTok{FALSE}\NormalTok{, }\AttributeTok{main =} \StringTok{""}\NormalTok{)}
\FunctionTok{curve}\NormalTok{(}\FunctionTok{dbeta}\NormalTok{(x, }\DecValTok{2}\NormalTok{, }\DecValTok{4}\NormalTok{), }\AttributeTok{col =} \DecValTok{2}\NormalTok{, }\AttributeTok{lwd =} \DecValTok{2}\NormalTok{, }\AttributeTok{add =} \ConstantTok{TRUE}\NormalTok{)}
\end{Highlighting}
\end{Shaded}

\includegraphics{Tarea2_EstadisticaII_files/figure-latex/unnamed-chunk-11-1.pdf}

Intervalo de credibilidad al 99\%

\begin{Shaded}
\begin{Highlighting}[]
\NormalTok{cred\_interval }\OtherTok{\textless{}{-}} \FunctionTok{quantile}\NormalTok{(valoresLambda, }\AttributeTok{probs =} \FunctionTok{c}\NormalTok{(}\FloatTok{0.005}\NormalTok{, }\FloatTok{0.995}\NormalTok{))}
\FunctionTok{cat}\NormalTok{(}\StringTok{"Intervalo de credibilidad al 99\%: ["}\NormalTok{, cred\_interval[}\DecValTok{1}\NormalTok{], }\StringTok{", "}\NormalTok{, cred\_interval[}\DecValTok{2}\NormalTok{], }\StringTok{"]}\SpecialCharTok{\textbackslash{}n}\StringTok{"}\NormalTok{)}
\end{Highlighting}
\end{Shaded}

\begin{verbatim}
## Intervalo de credibilidad al 99%: [ 0.01597998 ,  1.194903 ]
\end{verbatim}

Aceptacion o rechazo de lambda = 5

\begin{Shaded}
\begin{Highlighting}[]
\NormalTok{lambda\_hip }\OtherTok{\textless{}{-}} \FloatTok{0.5}
\ControlFlowTok{if}\NormalTok{(lambda\_hip }\SpecialCharTok{\textgreater{}=}\NormalTok{ cred\_interval[}\DecValTok{1}\NormalTok{] }\SpecialCharTok{\&\&}\NormalTok{ lambda\_hip }\SpecialCharTok{\textless{}=}\NormalTok{ cred\_interval[}\DecValTok{2}\NormalTok{]) \{}
  \FunctionTok{cat}\NormalTok{(}\StringTok{"No se rechaza la hipótesis lambda = 0.5, está dentro del intervalo de credibilidad.}\SpecialCharTok{\textbackslash{}n}\StringTok{"}\NormalTok{)}
\NormalTok{\} }\ControlFlowTok{else}\NormalTok{ \{}
  \FunctionTok{cat}\NormalTok{(}\StringTok{"Se rechaza la hipótesis lambda = 0.5, está fuera del intervalo de credibilidad.}\SpecialCharTok{\textbackslash{}n}\StringTok{"}\NormalTok{)}
\NormalTok{\}}
\end{Highlighting}
\end{Shaded}

\begin{verbatim}
## No se rechaza la hipótesis lambda = 0.5, está dentro del intervalo de credibilidad.
\end{verbatim}

\hypertarget{ejercicio-4}{%
\section{Ejercicio 4}\label{ejercicio-4}}

Funcion \[
f(x) = exp(\frac{sen(10x)}{10cos(x)})
\]

\begin{Shaded}
\begin{Highlighting}[]
\NormalTok{f }\OtherTok{\textless{}{-}} \ControlFlowTok{function}\NormalTok{(x) \{}
  \FunctionTok{return}\NormalTok{(}\FunctionTok{exp}\NormalTok{(}\FunctionTok{sin}\NormalTok{(}\DecValTok{10} \SpecialCharTok{*}\NormalTok{ x) }\SpecialCharTok{/}\NormalTok{ (}\DecValTok{10} \SpecialCharTok{*} \FunctionTok{cos}\NormalTok{(x))))}
\NormalTok{\}}
\end{Highlighting}
\end{Shaded}

Funcion de recalentamiento simulado

\begin{Shaded}
\begin{Highlighting}[]
\NormalTok{resim }\OtherTok{\textless{}{-}} \ControlFlowTok{function}\NormalTok{(f, }\AttributeTok{alpha =} \FloatTok{0.5}\NormalTok{, }\AttributeTok{s0 =} \DecValTok{5}\NormalTok{, }\AttributeTok{niter =} \DecValTok{1000}\NormalTok{, }\AttributeTok{mini =} \DecValTok{0}\NormalTok{, }\AttributeTok{maxi =} \DecValTok{10}\NormalTok{) \{}
\NormalTok{  s\_n }\OtherTok{\textless{}{-}}\NormalTok{ s0}
\NormalTok{  estados }\OtherTok{\textless{}{-}} \FunctionTok{rep}\NormalTok{(}\DecValTok{0}\NormalTok{, niter)}
\NormalTok{  iter\_count }\OtherTok{\textless{}{-}} \DecValTok{0}
  \ControlFlowTok{for}\NormalTok{ (k }\ControlFlowTok{in} \DecValTok{1}\SpecialCharTok{:}\NormalTok{niter) \{}
\NormalTok{    estados[k] }\OtherTok{\textless{}{-}}\NormalTok{ s\_n}
\NormalTok{    T }\OtherTok{\textless{}{-}}\NormalTok{ (}\DecValTok{1} \SpecialCharTok{{-}}\NormalTok{ alpha)}\SpecialCharTok{\^{}}\NormalTok{k  }\CommentTok{\# Enfriamiento}
\NormalTok{    s\_new }\OtherTok{\textless{}{-}} \FunctionTok{rnorm}\NormalTok{(}\DecValTok{1}\NormalTok{, s\_n, }\DecValTok{1}\NormalTok{)}
    
    \CommentTok{\# Asegurarse de que la nueva solución esté dentro de los límites}
    \ControlFlowTok{if}\NormalTok{ (s\_new }\SpecialCharTok{\textless{}}\NormalTok{ mini) \{ s\_new }\OtherTok{\textless{}{-}}\NormalTok{ mini \}}
    \ControlFlowTok{if}\NormalTok{ (s\_new }\SpecialCharTok{\textgreater{}}\NormalTok{ maxi) \{ s\_new }\OtherTok{\textless{}{-}}\NormalTok{ maxi \}}
    
\NormalTok{    dif }\OtherTok{\textless{}{-}} \FunctionTok{f}\NormalTok{(s\_new) }\SpecialCharTok{{-}} \FunctionTok{f}\NormalTok{(s\_n)}
    \ControlFlowTok{if}\NormalTok{ (dif }\SpecialCharTok{\textless{}} \DecValTok{0}\NormalTok{) \{}
\NormalTok{      s\_n }\OtherTok{\textless{}{-}}\NormalTok{ s\_new}
\NormalTok{    \} }\ControlFlowTok{else}\NormalTok{ \{}
\NormalTok{      random }\OtherTok{\textless{}{-}} \FunctionTok{runif}\NormalTok{(}\DecValTok{1}\NormalTok{, }\DecValTok{0}\NormalTok{, }\DecValTok{1}\NormalTok{)}
      \ControlFlowTok{if}\NormalTok{ (random }\SpecialCharTok{\textless{}} \FunctionTok{exp}\NormalTok{(}\SpecialCharTok{{-}}\NormalTok{dif }\SpecialCharTok{/}\NormalTok{ T)) \{}
\NormalTok{        s\_n }\OtherTok{\textless{}{-}}\NormalTok{ s\_new}
\NormalTok{      \}}
\NormalTok{    \}}
\NormalTok{    iter\_count }\OtherTok{\textless{}{-}}\NormalTok{ iter\_count }\SpecialCharTok{+} \DecValTok{1}
\NormalTok{  \}}
  \FunctionTok{return}\NormalTok{(}\FunctionTok{list}\NormalTok{(}\AttributeTok{r =}\NormalTok{ s\_n, }\AttributeTok{e =}\NormalTok{ estados))}
\NormalTok{\}}
\end{Highlighting}
\end{Shaded}

Aplicacion

\begin{Shaded}
\begin{Highlighting}[]
\NormalTok{Resultado }\OtherTok{\textless{}{-}} \FunctionTok{resim}\NormalTok{(f, }\FloatTok{0.1}\NormalTok{, }\DecValTok{5}\NormalTok{, }\DecValTok{1000}\NormalTok{, }\DecValTok{0}\NormalTok{, }\DecValTok{10}\NormalTok{)}
\end{Highlighting}
\end{Shaded}

Resultados

\begin{Shaded}
\begin{Highlighting}[]
\NormalTok{Resultado}\SpecialCharTok{$}\NormalTok{r  }\CommentTok{\# Minimo global}
\end{Highlighting}
\end{Shaded}

\begin{verbatim}
## [1] 4.711989
\end{verbatim}

\begin{Shaded}
\begin{Highlighting}[]
\FunctionTok{plot}\NormalTok{(Resultado}\SpecialCharTok{$}\NormalTok{e, }\AttributeTok{type =} \StringTok{"l"}\NormalTok{, }\AttributeTok{col =} \StringTok{"blue"}\NormalTok{, }\AttributeTok{lwd =} \DecValTok{2}\NormalTok{, }
     \AttributeTok{ylab =} \StringTok{"Estados"}\NormalTok{, }\AttributeTok{xlab =} \StringTok{"Iteraciones"}\NormalTok{, }\AttributeTok{main =} \StringTok{"Estados de la cadena"}\NormalTok{)}
\end{Highlighting}
\end{Shaded}

\includegraphics{Tarea2_EstadisticaII_files/figure-latex/unnamed-chunk-17-1.pdf}

\#Ejercicio 5 A)

\begin{Shaded}
\begin{Highlighting}[]
\CommentTok{\# Muestra}
\NormalTok{data }\OtherTok{\textless{}{-}} \FunctionTok{c}\NormalTok{(}\DecValTok{4}\NormalTok{, }\DecValTok{2}\NormalTok{, }\DecValTok{5}\NormalTok{, }\DecValTok{6}\NormalTok{, }\DecValTok{3}\NormalTok{, }\DecValTok{4}\NormalTok{, }\DecValTok{7}\NormalTok{, }\DecValTok{5}\NormalTok{, }\DecValTok{6}\NormalTok{, }\DecValTok{4}\NormalTok{)}
\CommentTok{\#Numero de iteraciones}
\NormalTok{n}\OtherTok{\textless{}{-}}\DecValTok{10000}
\CommentTok{\#Periodo quemado}
\NormalTok{L}\OtherTok{\textless{}{-}}\DecValTok{1000}
\CommentTok{\#Lambda artibtrario de inicio}
\NormalTok{lambda\_inicio}\OtherTok{\textless{}{-}}\FunctionTok{runif}\NormalTok{(}\DecValTok{1}\NormalTok{,}\DecValTok{0}\NormalTok{,}\DecValTok{10}\NormalTok{)}

\CommentTok{\# Función de verosimilitud de Poisson}
\NormalTok{poisson }\OtherTok{\textless{}{-}} \ControlFlowTok{function}\NormalTok{(lambda, data) \{}
  \FunctionTok{prod}\NormalTok{(}\FunctionTok{dpois}\NormalTok{(data, lambda))}
\NormalTok{\}}

\CommentTok{\# Función de distribución gamma a priori}
\NormalTok{gamma\_prior }\OtherTok{\textless{}{-}} \ControlFlowTok{function}\NormalTok{(lambda, alpha , beta ) \{}
  \FunctionTok{dgamma}\NormalTok{(lambda, }\AttributeTok{shape =}\NormalTok{ alpha, }\AttributeTok{rate =}\NormalTok{ beta)}
\NormalTok{\}}

\CommentTok{\# Algoritmo de Metropolis{-}Hastings}
\NormalTok{metropolis\_hastings }\OtherTok{\textless{}{-}} \ControlFlowTok{function}\NormalTok{(data, }\AttributeTok{N =}\NormalTok{ n, }\AttributeTok{lambda\_inicial =}\NormalTok{ lambda\_inicio, }
                                \AttributeTok{alpha =} \DecValTok{3}\NormalTok{, }\AttributeTok{beta =} \DecValTok{2}\NormalTok{) \{}
  
  
\NormalTok{  Intentos\_lambda }\OtherTok{\textless{}{-}} \FunctionTok{numeric}\NormalTok{(N)}
\NormalTok{  Intentos\_lambda[}\DecValTok{1}\NormalTok{] }\OtherTok{\textless{}{-}}\NormalTok{ lambda\_inicial}
\NormalTok{  lambda\_actual }\OtherTok{\textless{}{-}}\NormalTok{ lambda\_inicial}
\NormalTok{  Saltos }\OtherTok{\textless{}{-}} \DecValTok{0}
  
  \ControlFlowTok{for}\NormalTok{ (i }\ControlFlowTok{in} \DecValTok{2}\SpecialCharTok{:}\NormalTok{N) \{}
\NormalTok{    propuesta }\OtherTok{\textless{}{-}} \FunctionTok{rnorm}\NormalTok{(}\DecValTok{1}\NormalTok{, }\AttributeTok{mean =}\NormalTok{ lambda\_actual, }\AttributeTok{sd =} \FloatTok{0.5}\NormalTok{)  }\CommentTok{\# Propuesta}
    
    \ControlFlowTok{if}\NormalTok{ (propuesta }\SpecialCharTok{\textgreater{}} \DecValTok{0}\NormalTok{) \{  }\CommentTok{\# Para evitar valores negativos de lambda, pues es una poisson }
      
      \CommentTok{\#Usando el factor de bayes, la propuesta/la actual}
\NormalTok{      Aceptacion }\OtherTok{\textless{}{-}}\NormalTok{ (}\FunctionTok{poisson}\NormalTok{(propuesta, data) }\SpecialCharTok{*} \FunctionTok{gamma\_prior}\NormalTok{(propuesta, alpha, beta)) }\SpecialCharTok{/}
\NormalTok{                          (}\FunctionTok{poisson}\NormalTok{(lambda\_actual, data) }\SpecialCharTok{*} \FunctionTok{gamma\_prior}\NormalTok{(lambda\_actual, }
\NormalTok{                                                                      alpha, beta))}
      
      \CommentTok{\#Criterio de Aceptacion o Rechazo}
      \ControlFlowTok{if}\NormalTok{ (}\FunctionTok{runif}\NormalTok{(}\DecValTok{1}\NormalTok{) }\SpecialCharTok{\textless{}}\NormalTok{ Aceptacion) \{}
\NormalTok{        lambda\_actual }\OtherTok{\textless{}{-}}\NormalTok{ propuesta}
\NormalTok{        Saltos }\OtherTok{\textless{}{-}}\NormalTok{ Saltos }\SpecialCharTok{+} \DecValTok{1}
\NormalTok{      \}}
\NormalTok{    \}}
\NormalTok{    Intentos\_lambda[i] }\OtherTok{\textless{}{-}}\NormalTok{ lambda\_actual}
\NormalTok{  \}}
  \FunctionTok{return}\NormalTok{(}\FunctionTok{list}\NormalTok{(}\AttributeTok{Intentos\_lambda =}\NormalTok{ Intentos\_lambda[(L}\SpecialCharTok{+}\DecValTok{1}\NormalTok{)}\SpecialCharTok{:}\NormalTok{N], }\AttributeTok{Saltos =}\NormalTok{ Saltos))}
\NormalTok{\}}

\CommentTok{\# Ejecutar el algoritmo con n = 10000}
\NormalTok{Muestras\_MCMC }\OtherTok{\textless{}{-}} \FunctionTok{metropolis\_hastings}\NormalTok{(data)}\SpecialCharTok{$}\NormalTok{Intentos\_lambda}
\end{Highlighting}
\end{Shaded}

\begin{enumerate}
\def\labelenumi{\Alph{enumi})}
\setcounter{enumi}{1}
\tightlist
\item
\end{enumerate}

\begin{Shaded}
\begin{Highlighting}[]
\FunctionTok{hist}\NormalTok{(Muestras\_MCMC, }\AttributeTok{breaks =} \DecValTok{30}\NormalTok{, }\AttributeTok{prob =} \ConstantTok{TRUE}\NormalTok{, }\AttributeTok{main =} \StringTok{"Histograma de la muestra MCMC"}\NormalTok{,}
     \AttributeTok{xlab =} \FunctionTok{expression}\NormalTok{(lambda),}\AttributeTok{ylab =} \StringTok{"Distribución"}\NormalTok{, }\AttributeTok{col =} \StringTok{"blue"}\NormalTok{)}
\end{Highlighting}
\end{Shaded}

\includegraphics{Tarea2_EstadisticaII_files/figure-latex/unnamed-chunk-19-1.pdf}

\begin{enumerate}
\def\labelenumi{\Alph{enumi})}
\setcounter{enumi}{2}
\tightlist
\item
\end{enumerate}

\begin{Shaded}
\begin{Highlighting}[]
\FunctionTok{plot}\NormalTok{(Muestras\_MCMC, }\AttributeTok{type =} \StringTok{"l"}\NormalTok{, }\AttributeTok{main =} \StringTok{"Traceplot de la muestra MCMC"}\NormalTok{,}
     \AttributeTok{xlab =} \StringTok{"Iteraciones"}\NormalTok{, }\AttributeTok{ylab =} \FunctionTok{expression}\NormalTok{(lambda), }\AttributeTok{col =} \StringTok{"blue"}\NormalTok{)}
\end{Highlighting}
\end{Shaded}

\includegraphics{Tarea2_EstadisticaII_files/figure-latex/unnamed-chunk-20-1.pdf}

\begin{enumerate}
\def\labelenumi{\Alph{enumi})}
\setcounter{enumi}{3}
\tightlist
\item
\end{enumerate}

\begin{Shaded}
\begin{Highlighting}[]
\FunctionTok{acf}\NormalTok{(Muestras\_MCMC, }\AttributeTok{main =} \StringTok{"Gráfico de Autocorrelación de la muestra MCMC"}\NormalTok{)}
\end{Highlighting}
\end{Shaded}

\includegraphics{Tarea2_EstadisticaII_files/figure-latex/unnamed-chunk-21-1.pdf}

\begin{enumerate}
\def\labelenumi{\Alph{enumi})}
\setcounter{enumi}{4}
\tightlist
\item
\end{enumerate}

\begin{Shaded}
\begin{Highlighting}[]
\CommentTok{\#Normalmente es la suma acumulada/(Iteracion{-}periodo\_quemado)}

\NormalTok{mean\_muestras }\OtherTok{\textless{}{-}} \FunctionTok{cumsum}\NormalTok{(Muestras\_MCMC) }\SpecialCharTok{/} \FunctionTok{seq\_along}\NormalTok{(Muestras\_MCMC)}
\FunctionTok{plot}\NormalTok{(mean\_muestras, }\AttributeTok{type =} \StringTok{"l"}\NormalTok{, }\AttributeTok{main =} \StringTok{"Convergencia ergódica de la media de la muestra MCMC"}\NormalTok{,}
     \AttributeTok{xlab =} \StringTok{"Iteraciones"}\NormalTok{, }\AttributeTok{ylab =} \StringTok{"Promedios acumulados"}\NormalTok{, }\AttributeTok{col =} \StringTok{"blue"}\NormalTok{)}
\end{Highlighting}
\end{Shaded}

\includegraphics{Tarea2_EstadisticaII_files/figure-latex/unnamed-chunk-22-1.pdf}

\begin{enumerate}
\def\labelenumi{\Alph{enumi})}
\setcounter{enumi}{5}
\tightlist
\item
\end{enumerate}

\begin{Shaded}
\begin{Highlighting}[]
\NormalTok{mean\_lambda }\OtherTok{\textless{}{-}} \FunctionTok{mean}\NormalTok{(Muestras\_MCMC)}
\FunctionTok{cat}\NormalTok{(}\StringTok{"Estimación de lambda:"}\NormalTok{, mean\_lambda, }\StringTok{"}\SpecialCharTok{\textbackslash{}n}\StringTok{"}\NormalTok{)}
\end{Highlighting}
\end{Shaded}

\begin{verbatim}
## Estimación de lambda: 4.099539
\end{verbatim}

\begin{Shaded}
\begin{Highlighting}[]
\FunctionTok{cat}\NormalTok{(}\StringTok{"Tasa de aceptación }\SpecialCharTok{\textbackslash{}n}\StringTok{"}\NormalTok{,}\StringTok{"NumeroSaltos/TotalIteraciones:"}\NormalTok{ , }
\NormalTok{    (}\FunctionTok{metropolis\_hastings}\NormalTok{(data)}\SpecialCharTok{$}\NormalTok{Saltos)}\SpecialCharTok{/}\NormalTok{(n}\SpecialCharTok{{-}}\NormalTok{L) ,}\StringTok{"}\SpecialCharTok{\textbackslash{}n}\StringTok{"}\NormalTok{)}
\end{Highlighting}
\end{Shaded}

\begin{verbatim}
## Tasa de aceptación 
##  NumeroSaltos/TotalIteraciones: 0.8252222
\end{verbatim}

\end{document}
